\documentclass[12pt,a4paper,oneside]{book}
\usepackage{amsmath, amssymb, fancyhdr, wasysym,tabvar}
\usepackage[utf8]{vietnam}
\usepackage{tkz-euclide,tikz-3dplot, tikz,tkz-tab}
\usepackage[many]{tcolorbox}
\usepackage{pgf}
\usetikzlibrary{shapes.geometric,arrows,snakes,calc,intersections,angles}
\usepackage[tikz]{bclogo} 
\usepackage{mathpazo}
\usepackage{fancyhdr}
\pagestyle{fancy}
\usepackage{pgfplots}
\usepgfplotslibrary{fillbetween}
\usetikzlibrary{patterns}
\pgfplotsset{compat=1.9}
\usepackage[top=1cm, bottom=1.5cm, left=1.5cm, right=1.5cm] {geometry}
\usepackage[hidelinks,unicode]{hyperref}
\usepackage{esvect}
\def\vec{\vv}
\def\overrightarrow{\vv}
%Lệnh của gói mathrsfs
\DeclareSymbolFont{rsfs}{U}{rsfs}{m}{n}
\DeclareSymbolFontAlphabet{\mathscr}{rsfs}
%Lệnh cung
\DeclareSymbolFont{largesymbols}{OMX}{yhex}{m}{n}
\DeclareMathAccent{\wideparen}{\mathord}{largesymbols}{"F3}
%Lệnh song song
\DeclareSymbolFont{symbolsC}{U}{txsyc}{m}{n}
\DeclareMathSymbol{\varparallel}{\mathrel}{symbolsC}{9}
\DeclareMathSymbol{\parallel}{\mathrel}{symbolsC}{9}
%Hệ hoặc, hệ và
\newcommand{\hoac}[1]{ 
\left[\begin{aligned}#1\end{aligned}\right.}
\newcommand{\heva}[1]{
\left\{\begin{aligned}#1\end{aligned}\right.}
% định nghĩa tập hợp các số
\newcommand{\R}{\mathbb R}
%%%%%%%%%%%%%%%%%%%%%Lời giải%%%%%%%%%%%%%%%%%%%%%%%%%%%%%%%5
%\usepackage[loigiai]{ex_test}%dethi,loigiai,color,solcolor,book
%%%%%%%%%%%%%%%%%%%%%%Đề thi%%%%%%%%%%%%%%%%%%%%%%%%%%%%%%%%%%
\usepackage[dethi]{ex_test}
%==========================================================
\newenvironment{dong}[1]{%
    \begin{center}
        \textbf{\underline{Bài giải:}}\\[0.5cm] % Centered "Bài giải:" in bold and underlined
        \foreach \i in {1,...,#1} {%
            \makebox[16.5cm]{\dotfill}\\[0.6cm] % Dotted line with adjustable spacing
        }%
    \end{center}
}{}
\fancyfoot[C]{%
  \makebox[\textwidth][c]{\textbf{\textcolor{black}{Trang \thepage}}}%
}

\renewcommand{\headrulewidth}{0pt} % Tùy chọn: bỏ gạch ngang ở header
\renewcommand{\footrulewidth}{0pt} % Tùy chọn: bỏ gạch ngang ở footer
%%%%%%%%%%%%%%%%%%%%%%%%%%%%%%%%%%%%%%%%%%%%%%%%%%%%%%%%%%%%
\begin{document}
\begin{center}
\begin{tabular}{ccc}
{\Large\textbf{BKC EDUCATION}}&$\hspace{2.5cm}$& {\Large\textbf{ĐỀ ÔN TỐT NGHIỆP THPT -- 2025}}\\
\boxed{\text{ĐÊ ÔN SỐ 36}} && \textbf{\textit{NĂM HỌC}:} 2024 -- 2025 \\
&& \textbf{MÔN:} TOÁN -- \textbf{LỚP:} 12 
\end{tabular}
\end{center}
\begin{tabular}{cc}
 \textbf{Họ, tên học sinh:}.......................................................................... &$\hspace{4cm}$    \boxed{\text{MÃ ĐỀ: 36}}
\end{tabular}\\
\noindent\textbf{PHẦN I.} Thí sinh trả lời từ câu 1 đến câu 12. Mỗi câu hỏi thí sinh chỉ lựa chọn một phương án. \\
\Opensolutionfile{ans}[ans/ansTN01]
%\input{data/TF.tex}
\input{data/TN}
\Closesolutionfile{ans}
\Opensolutionfile{ans}[ans/ansTF01]
\noindent\textbf{PHẦN II.}
Thí sinh trả lời từ câu 1 đến câu 4. Trong mỗi ý \textbf{a), b), c) d)} ở mỗi câu, thí sinh chọn đúng hoặc sai.\\
\setcounter{ex}{0}
\input{data/TF}

%\Closesolutionfile{ans}
%\Opensolutionfile{ans}[ans/ansShort01]
\noindent\textbf{PHẦN III.} Thí sinh trả lời câu 1 đến câu 6.
%\Opensolutionfile{ans}[ans/ans-cau-KQ-1]
\setcounter{ex}{0}
\input{data/short}
%\Closesolutionfile{ans}
\newpage
\begin{center}
\textbf{ĐÁP ÁN}
\end{center}
% \inputans{6}{ans/ansTN01}
% \inputans{4}{ans/ansTF01}
% \inputans{6}{ans/ansShort01}
\end{document}